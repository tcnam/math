\documentclass{article}
\usepackage[utf8]{vietnam}
\usepackage{graphicx}
\usepackage{amssymb}
\usepackage{titlesec}
\usepackage{lipsum}

\graphicspath{ {./images/} }
\newcommand\tab[1][1cm]{\hspace*{#1}}

\title{DẶM ĐƯỜNG TÔI ĐI}

\begin{document}
\maketitle

\tab Quyển sách là cuốn tự truyện của ông Võ Quang Huệ. Ông tốt nghiệp chuyên ngành ô tô và động cơ ô tô tại Đức.
Ông Huệ có 24 năm làm việc tại BMW trong bộ phận nghiên cứu phát triển xe ô tô mới và lãnh đạo các đề án
sản xuất - kinh doanh tại các thị trường Đông Nam Á, Mexico và Ai cập. Ông Huệ là tổng giám đốc đầu tiên của công 
ty Bosch tại Việt Nam (2007-2017), ông là một trong những người thuyết phục Bosch mở nhà máy ở Long Thành và trung tâm
RandD ở Việt Nam, cựu phó tổng giám đốc tập đoàn Vingroup, phụ trách chuyên đề Vinfast.\\

\tab Ông sinh năm 1952 tại huyện Điện Bàn, phía bắc tỉnh Quảng Nam. Sinh ra trong một gia đình khá giả thuộc nhóm 
'ăn cơm quốc gia thờ ma cộng sản'. Từ nhỏ ông đã được cha mình dạy dỗ và định hướng rất chỉnh chu. Cha cho ông học 
tiếng Pháp để đến châu Âu thay vì tiếng Anh như những bạn bè đồng trang lứa ở miền Nam. Tuy gia đình khá giả nhưng 
khi ông thích mua gì, ba ông chỉ cho tối đa 2/3 số tiền, còn lại phải tự đi góp phần còn lại. Để kiếm tiền, 
ông chọn cách vào mùa hè làm việc trong xưởng hồ và dệt của gia đình. Nhờ thế ngay từ nhỏ, ông đã học được tính cách 
năng động và dễ dàng thích ứng. Từ khi còn nhỏ, ông đã thể hiện một đam mê to lớn với xe cộ cộng với gia đình khá giả
ông dấn thân vào con đường độ xe, chơi xe, một thú vui xa xỉ tại thời điểm đó. \\
\tab Năm 1970, chính quyền Việt Nam cộng hòa bắt đầu không cho du học đến Pháp, vì có nhiều học sinh đến Pháp theo 
phong trào hòa bình mạnh mẽ. Sẵn đam mê với ô tô, ông quyết định chọn du học Đức - nơi có nền công nghiệp ô tô dẫn 
đầu thế giới. Trong thời gian sống cùng một gia đình Đức, ông được họ dạy về 2 đức tính: kỉ luật và đúng giờ.\\
\tab Trong quãng thời gian học đại học ở đây, ông tích cực tham gia các phong trào biểu tình, phản đối chiến tranh ở 
Việt Nam.\\
\tab Sau giải phóng 30/4/1975, gia đình ông bị xếp vào diện tư sản. Căn nhà của gia đình ông ở SG bị trưng dụng làm 
trụ sở công an phường. Gia đình bị bắt đi kinh tế mới. Chỉ mãi đến khi về nước 1979, ông mới biết nhiều điều này, ba
ông khuyên ông nên tiếp tục ở lại Đức học tập, đừng vội hồi hương. Sau chuyến thăm này, tổ chức yêu nước của sinh viên
ở Đức bắt đầu có sự chia rẽ.\\

\tab \textbf{REVIEW TIẾP TỤC NGÀY 27/10/2024}\\
\tab \textbf{Khoảng thời gian ở BMW}\\
\tab Sau khi trở về châu Âu, ông tập trung hoàn toàn vào phát triển sự nghiệp cá nhân. Ông làm thêm công việc trợ lý
tại Viện Ứng dụng nhiệt động học ở Aachen của giáo sư Franz Pischinger (một nhân vật nổi tiếng trong ngành ô tô). 
Sau khi tốt nghiệp, dù được nhận vào nhiều công ty (BMW, Mercedes Benz, Ford, Volkswagen,...). Nhưng sau cùng vẫn
quyết định chọn BMW. Tại đây, ông làm việc tại nhiều phòng ban thuộc mảng R\&D như\\ 
\tab\tab - Tiếng động của động cơ \\
\tab\tab - Thử nghiệm lái xe khắp mọi địa hình, tốc độ\\
\tab\tab - Nâng cao hiệu suất động cơ\\
\tab\tab - Giảm hao tốn nhiên liệu, giảm khí thải cho xe\\
\tab\tab - Tham gia phát triển dòng xe mới với vai trò leader. Ông làm việc với rất nhiều bộ phận liên quan marketing,
kinh doanh, R\&D, sản xuất, sửa chữa, dịch vụ khách hàng, các bên thứ 3 cung cấp linh kiện. \\
\tab Sau đó, ông được điều chuyển sang làm ở mảng sản xuất ở các thị trường mới. Thị trường đầu tiên là Việt Nam (1994), 
lắp ráp xe BMW theo dạng SKD (Semi knocked down - lắp xe từ những cụm linh kiện đã được lắp từ trước) 
hợp tác giữa BMW và công ty ô tô Hòa Bình (VMC-Vietnam Motors Corporation). Sau thành công ở Việt Nam, ông được điều về điều 
về bộ phận sản xuất khu vực Đông Nam Á. Tiếp theo là thị trường Mexico, tại đây tìnhh hình an ninh chính trị không được đảm 
bảo lắm nên hầu như các mẫu xe sang sản xuất ở đây phải có thêm tính chống đạn. Một số sự cố xảy ra như mất tiền công ty hay
đồng nghiệp bị bắt cóc. Sau đó, ông và gia đình có một khoảng thời gian work life balance ở Ai Cập\\
\tab \textbf{Khoảng thời gian ở Bosch}\\
\tab Sau khoảng thời gian yên bình ở Ai Cập, ông có dự định nghỉ BMW để về VN, mở cty riêng, thì được Bosch offer về làm ban
quản lý cho công ty con ở VN (mở ra để bán hàng ở VN- tức là chỉ mảng kinh doanh). Nhưng sau đó nhờ những nỗ lực thuyết phục
không ngừng nghỉ, ông đã thuyết phục được các lãnh đạo cấp cao của Bosch xây dựng nhà máy sản xuất dây đai truyền lực cho hộp
số tự động (ý định lúc đầu là ở Trung Quốc), mở trung tâm R\&D (Bosch quận 1), trung tâm công nghệ phần mềm (Bosch Tân Bình).
Hiểu nôm na, ông gần như đã đem end-to-end Bosch về VN, từ khâu R\&D đến sản xuất phần cứng, tời phần mềm cho phần cứng
tới bán hàng, tạo công ăn việc làm cho biết bao nhiêu người cũng như làm tăng nhu cầu của các ngành công nghiệp, dịch vụ phụ trợ\\
\tab Trong khoảng thời gian này ông vẫn luôn tuân thủ tuyệt đối những quy định của công ty, không có zụ đi của sau hay lách luật, 
trốn thuế (như các công ty VN).\\

\tab \textbf{REVIEW TIẾP TỤC NGÀY 22/11/2024}\\
\tab \textbf{Khoảng thời gian ở Vinfast}\\
\tab Trong buổi đầu tiên gặp mặt, ông Phạm Nhật Vượng đã đưa ra bài toán muốn làm được ô tô trong vòng 2 năm.
Theo ý kiến của ông Huệ việc ra xe trong 2 năm là một việc bất khả thi, đến cả những tập đoàn lâu đời như BMW
cũng phải mất tới 4 đến 5 năm nếu muốn tung ra một mẫu xe mới. Trong khi đó nền công nghiệp ô tô ở Việt Nam là 
con số 0, thiếu vắng hoàn toàn các ngành công nghiệp phụ trợ, thiếu nguồn nhân lực, thiếu kinh nghiệm,\dots 
Thế nhưng không phải là không khả thi, ông Huệ đưa ra cả giải pháp ngắn hạn để ra xe kịp trong 2 năm cũng như 
giải pháp dài hạn giúp VN thật sự thiết kế được một chiếc xe.\\
\tab Giải pháp ngắn hạn:\\
\tab - Tìm kiếm một nhà phân phối động cơ và gầm xe (2 bộ phận phức tạp nhất trên ô tô, chiếm 60-70 phần trăm
công sức tạo nên 1 chiếc xe, nền móng cho tất cả công đoạn còn lại). Đối tác được lựa chọn là BMW (ông hiểu rõ cách thức hoạt động của BMW).\\
\tab - Tìm kiếm nhà thiết kế xe (nội thất sao, vẻ ngoài sao, \dots). Những cái tên được chọn là Magna Steyr, AVL.\\
\tab - Mua thiết bị lắp ráp để xe xuất xưởng đạt chuẩn.\\
\tab - Cho ra mắt trước xe máy điện để khiến thị trường biết tới cái tên Vinfast.\\
\tab Giả pháp dài hạn:\\
\tab - Xây dựng trường đạo tạo nghề ngay trong nhà máy hướng tới hướng học đi đôi với hành. Đảm bảo luôn có 
nguồn nhân lực mới trong trường hợp nhân lực cũ nghỉ (cái này hay vãi).\\
\tab - Xây dựng trung tâm R\&D để đào tạo đội ngũ tri thức cao, từng bước làm chủ công nghệ.\\
\tab - Theo sát các chuyên gia nước ngoài trong dự án này để học tập kinh nghiệp, quy trình, công nghệ trong industry.\\
\tab - Đầu tư tiền của vào nền công nghiệp phụ trợ.\\
\tab - Tìm các đối tác để liên doanh nhằm làm chủ công nghệ pin trên xe điện.\\
\tab Kết thúc buổi nói chuyện, ông Vượng hỏi ông Huệ trong 2 vị trí, một là CEO công ty ô tô mới (vinfast), hai là 
phó tổng giám đốc Vingroup, ông chọn vị trí nào. Ông Huệ chọn vị trí phó tổng giám đốc Vingroup do vị trí còn lại 
ông không thể gắn bó được lâu dài. Ngay sáng hôm sau, ông Huệ nhận được offer letter của Vingroup.\\
\tab Khoảng thời gian tiếp theo một chuỗi ngày bận rộn. Các giải pháp đều yêu cầu khối lượng thương thảo rất lớn.
Trong đó cái quan trọng nhất là lấy được giấy phép bản quyền sử dụng động cơ và gầm xe của BMW. Trong buổi thương 
thảo, BMW đưa ra một loạt câu hỏi rất khó trả lời (kế hoạch cụ thể, định hướng, thị trường nhắm tới, \dots). Nếu mọi 
chuyện cứ tiếp diễn như thế này cuộc thương thảo sẽ thất bại, thế là ông đề nghị giải lao để nói chuyện riêng vs đại 
diện của BMW. Ông đưa ra một vài ý chính: \\
\tab - Vingroup có tiền chỉ cần BMW đồng ý là thu tiền ngay lập tức, nếu thương vụ thành công sẽ giúp đại diện của BMW 
thăng tiến trong sự nghiệp.\\
\tab - Ông Huệ có hơn 20 năm làm trong BMW và trải qua rất nhiều vị trí nên có thể đảm bảo chất lượng đầu ra của xe Vinfast.\\
\tab - Chỉ ra đại diện bên phía BMW hiện tại cũng chỉ là người đi làm thuê nên khi nhìn vào bản kế hoạch của Vinfast sẽ 
không tự tin sẽ thuyết phục được những lãnh đạo cấp cao của BMW.\\
\tab - Nếu đại diện BMW đồng ý thì sẽ có được một đối tác là ông Huệ người sẽ cùng ông thuyết phục các ban lãnh đạo cấp 
cao của BMW.\\
\tab Việc tuyển nhân sự diễn ra khá suôn sẻ.\\
\tab Sau khi mọi thứ đã vào guồng với dự án xe xăng, ông lập tức đi thương thảo về công nghệ pin, xe điện. Mỗi chuỗi ngày liên
tục kéo dài suốt 28 ngày đi đến 18 thành phố 7 quốc gia khác nhau. Giữa các chuyến bay, ông còn phải ngồi soạn và gửi báo cáo 
đến cho ông Vượng. Ông tự lái xe đến địa điểm họp để tiết kiệm thời gian nhiều nhất có thể.\\
\tab Sau khi ra mắt 2 chiếc xe xăng, ông Huệ xin từ chức và chỉ tham gia với vai trò cố vấn. Kết thúc một chặng đường cháy hết 
mình.\\



\end{document}