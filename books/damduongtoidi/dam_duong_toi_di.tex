\documentclass{article}
\usepackage[utf8]{vietnam}
\usepackage{graphicx}
\usepackage{amssymb}
\usepackage{titlesec}
\usepackage{lipsum}

\graphicspath{ {./images/} }
\newcommand\tab[1][1cm]{\hspace*{#1}}

\title{DẶM ĐƯỜNG TÔI ĐI}

\begin{document}
\maketitle

\tab Quyển sách là cuốn tự truyện của ông Võ Quang Huệ. Ông tốt nghiệp chuyên ngành ô tô và động cơ ô tô tại Đức.
Ông Huệ có 24 năm làm việc tại BMW trong bộ phận nghiên cứu phát triển xe ô tô mới và lãnh đạo các đề án
sản xuất - kinh doanh tại các thị trường Đông Nam Á, Mexico và Ai cập. Ông Huệ là tổng giám đốc đầu tiên của công 
ty Bosch tại Việt Nam (2007-2017), ông là một trong những người thuyết phục Bosch mở nhà máy ở Long Thành và trung tâm
RandD ở Việt Nam, cựu phó tổng giám đốc tập đoàn Vingroup, phụ trách chuyên đề Vinfast.\\

\tab Ông sinh năm 1952 tại huyện Điện Bàn, phía bắc tỉnh Quảng Nam. Sinh ra trong một gia đình khá giả thuộc nhóm 
'ăn cơm quốc gia thờ ma cộng sản'. Từ nhỏ ông đã được cha mình dạy dỗ và định hướng rất chỉnh chu. Cha cho ông học 
tiếng Pháp để đến châu Âu thay vì tiếng Anh như những bạn bè đồng trang lứa ở miền Nam. Tuy gia đình khá giả nhưng 
khi ông thích mua gì, ba ông chỉ cho tối đa 2/3 số tiền, còn lại phải tự đi góp phần còn lại. Để kiếm tiền, 
ông chọn cách vào mùa hè làm việc trong xưởng hồ và dệt của gia đình. Nhờ thế ngay từ nhỏ, ông đã học được tính cách 
năng động và dễ dàng thích ứng. Từ khi còn nhỏ, ông đã thể hiện một đam mê to lớn với xe cộ cộng với gia đình khá giả
ông dấn thân vào con đường độ xe, chơi xe, một thú vui xa xỉ tại thời điểm đó. \\
\tab Năm 1970, chính quyền Việt Nam cộng hòa bắt đầu không cho du học đến Pháp, vì có nhiều học sinh đến Pháp theo 
phong trào hòa bình mạnh mẽ. Sẵn đam mê với ô tô, ông quyết định chọn du học Đức - nơi có nền công nghiệp ô tô dẫn 
đầu thế giới. Trong thời gian sống cùng một gia đình Đức, ông được họ dạy về 2 đức tính: kỉ luật và đúng giờ.\\
\tab Trong quãng thời gian học đại học ở đây, ông tích cực tham gia các phong trào biểu tình, phản đối chiến tranh ở 
Việt Nam.\\
\tab Sau giải phóng 30/4/1975, gia đình ông bị xếp vào diện tư sản. Căn nhà của gia đình ông ở SG bị trưng dụng làm 
trụ sở công an phường. Gia đình bị bắt đi kinh tế mới. Chỉ mãi đến khi về nước 1979, ông mới biết nhiều điều này, ba
ông khuyên ông nên tiếp tục ở lại Đức học tập, đừng vội hồi hương. Sau chuyến thăm này, tổ chức yêu nước của sinh viên
ở Đức bắt đầu có sự chia rẽ


\end{document}