\documentclass{article}
\usepackage[utf8]{vietnam}
\usepackage{graphicx}
\usepackage{amssymb}
\usepackage{titlesec}
\usepackage{lipsum}

\graphicspath{ {./images/} }
\newcommand\tab[1][1cm]{\hspace*{#1}}

\title{THẢM HỌA KHÍ HẬU}

\begin{document}
\maketitle

23-Dec-2024


Có 2 con số cần biết về biến đổi khí hậu. Một là 51 tỷ (số tấn khí thải gây hiệu ứng nhà kính được xả ra mỗi năm)
, con số còn lại là 0 (mục tiêu cần hướng tới)\\

\section{Tại sao lại là số 0}

\tab Con số 0 không phải là ngừng xả thải mà là đạt được trạng thái trung hòa carbon.\\
\tab Mặc dù nhiệt độ trung bình chỉ tăng 1 độ C nhưng thực tế mức tăng ở một số nơi đã là hơn 2 độ C. Và những 
khu vực này là nơi sinh sống của từ 20 đến 40 phần trăm dân số.\\
\tab Khí nhà kính bao gồm một số như Carbon dioxit, nito oxit, methane. Để đo lường mức dộ ảnh hưởng của các loại 
khí nhà kính, người ta dùng một đơn vị chung "carbon dioxit quy đổi". Nguyên nhân tại sao các loại khí này gây ra 
hiện tượng nhà kính là do: khi một loại phân tử nhất định bị một bức xạ ở bước sóng nhất định tác động, chúng sẽ chặn
bức xạ, hấp thụ năng lượng của nó và dao động nhanh hơn. Dẫn tới nhiệt độ tăng cao. Các loại khí chỉ được cấu tạo 
từ một chất như O2, N2 thì cho phép ánh sáng trắng đi qua nên không làm tăng nhiệt độ.\\
\tab Hậu quả của việc Trái Đất nóng lên.\\
\tab\tab - Tần suất, cường độ thời tiết khắt nhiệt tăng lên.\\
\tab\tab - Gây nên hạn hán, cháy rừng. Do không khí nóng hơn nên nó sẽ lấy thêm nước từ đất.\\
\tab\tab - Mực nước biển tăng cao.\\
\tab\tab - Giảm sản lượng nông nghiệp, chăn nuôi. Mất an ninh lương thực.\\
\tab\tab - Phá hủy hệ sinh thái, nhiều loài động thực vật chết hoặc di cư sang nơi khác -> tạo nên các loại dịch bệnh 
mới.\\

\section{Một hành trình khó khăn}
\tab Việc thay thế các nguồn gây phát thải khí nhà kính thực sự rất khó khăn, một số nguyên nhân như:\\
\tab\tab - Tất cả các sản phẩm hiện nay từ bàn chải đánh răng cho đến xi măng, bánh mì,\dots Quy trình sản xuất đều cần 
có sự tham gia của nguyên liệu hóa thạch. Thể giới sử dụng hơn 15 tỷ lít dầu mỗi ngày \\
\tab\tab - Dầu thô rẻ hơn cả nước giải khát. Trong năm 2020, một thùng dầu 160 lít giá 42 đô la\\
\tab\tab - Mức sống ngày một nâng cao dẫn đến nhu cầu tiêu thụ năng lượng càng tăng cao\\
\tab\tab - Việc chuyển dịch từ động cơ đốt trong - sử dụng nguyên liệu hóa thạch, sang động cơ sử dụng nguồn nguyên liệu 
sạch cần rất nhiều thời gian.\\
\tab\tab - Quy trình phức tạp trong việc thương mại hóa các động cơ, thiết bị sử dụng nguyên liệu sạch\\
\tab\tab - Xã hội chấp nhận ít rủi ro trong kinh doanh năng lượng, do bản thân năng lượng sạch vẫn chưa thật sự đủ ổn định\\
\tab\tab - Luật pháp quá lỗi thời, chính sách thay đổi khi đảng khác lên nắm quyền\\
\tab\tab - Cần sự đồng thuận ở cấp độ quốc gia để tiến hành các chính sách liên quan đến môi trường \\
\end{document}