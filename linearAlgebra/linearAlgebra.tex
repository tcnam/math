\documentclass{article}
\usepackage[utf8]{vietnam}
\usepackage{amssymb}
\usepackage{titlesec}

\setcounter{secnumdepth}{4}
\titleformat{\paragraph}
{\normalfont\normalsize\bfseries}{\theparagraph}{1em}{}
\titlespacing*{\paragraph}
{0pt}{3.25ex plus 1ex minus .2ex}{1.5ex plus .2ex}


\title{ĐẠI SỐ TUYẾN TÍNH}
\author{Trần Công Nam}

\begin{document}
\maketitle

\part[short]{Logic, tập hợp, ánh xạ, số phức}

\section[short]{Đại cương về logic}

\subsection{Mệnh đề và các phép toán}
\subsubsection {Định nghĩa}
\begin{tabbing}
- Định nghĩa: là một khẳng định chỉ có thể đúng hoặc sai\\
- Kí hiệu: chữ cái in hoa A, B, \dots \\
- \= Giá trị chân lý: \\
\> + A đúng = 1 \\
\> + A sai = 0  \\
VD1: A = "3>1" = 1 \\
VD2: B="Việt Nam thuộc châu Âu" = 0 \\
\end{tabbing}

\subsubsection {Các phép toán về mệnh đề}

\paragraph{Phép phủ định}
- Kí hiệu $\overline{A}$\\

\paragraph{Phép hội}
- Định nghĩa: hội của 2 mệnh đề A và B là "A và B"\\
- Kí hiệu: $A \wedge B$\\ 




\part{Ma trận, định thức, hệ phương trình}

\part{Không gian vecto}

\part{Ánh xạ tuyến tính}

\part{Dạng toàn phương và không gian Euclid}


\end{document}