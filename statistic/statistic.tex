\documentclass{article}
\usepackage[utf8]{vietnam}
\usepackage{amssymb}

\title{XÁC XUẤT THỐNG KÊ}
\author{Trần Công Nam}

\begin{document}
\maketitle

\part{Sự kiện ngẫu nhiên và phép tính xác suất}

\section{Sự kiện và các phép toán của sự kiện}
\subsection{Phép thử ngẫu nhiên và sự kiện sơ cấp}
- Định nghĩa: là hành động mà kết quả của nó ngẫu nhiên, nhưng ta xác định được tập hợp kết quả có thể.\\
- Tập hợp kết quả của phép thử là \textbf{không gian mẫu}, kí hiệu: $\Omega$.\\
- Mỗi phần tử trong tập hợp $\Omega$ là một \textbf{sự kiện sơ cấp}.\\
- VD1: Tung 1 con xúc xắc và quan sát mặt xuất hiện.\\
+ Phép thử: tung 1 con xúc xắc.\\
+ không gian mẫu: $\Omega$ = \{$w_1$, $w_2$, $w_3$, $w_4$, $w_5$, $w_6$\}, trong đó $w_i$ là sự kiện sơ cấp "mặt i chấm xuất hiện" i = 1,6.\\
- VD2: Xét phép thử kiểm tra tuổi thọ của một bóng đèn\\
+ Không gian mẫu: $\Omega=\{x: x\geq 0\}$\\

\subsection{Sự kiện}
- Định nghĩa: là tập hợp các sự kiện sơ cấp.\\
- Kí hiệu: chữ cái in hoa \\
- VD1: Giả sử A là một sự kiện của phép thử. Sự kiện sơ cấp w của phép thử được gọi là \textbf{thuận lợi cho A}. Kí hiệu: $x \in A$ nếu w xảy ra thì A xảy ra.\\
- Nhận xét: Néu A là 1 sự kiện của phép thử thì\\
+ A là tập con của $\Omega$ \\
+ A = \{sự kiện sơ cấp\} \\
- Có 2 sự kiện đặc biệt \\
+ Sự kiện không thể có: $\emptyset$ \\
+ Sự kiện chắc chắn: $\Omega$ \\

\subsection{Quan hệ và phép toán cuả các sự kiện}

\subsubsection {Quan hệ kéo theo}
- Định nghĩa: Sự kiện A được gọi là kéo theo sự kiện B, nếu khi A xảy ra thì B xảy ra hay \textbf{A là con B}, kí hiệu: $A \subseteq B$ \\
- VD1: Sinh viên mua một tờ vé số. \\
    + A: "sv có vé số trúng giải đặc biệt" \\
    + B: "sv có vé số trúng giải" \\
    -> Ta nói $A \subseteq B$ \\
\subsubsection {Quan hệ tương đương}
- Định nghĩa: 2 sự kiện A và B được gọi là tương đương, khi tập hợp sự kiện sơ cấp của A và B là như nhau, kí hiệu: A = B\\

\subsubsection[short]{Hợp}
- Định nghĩa: Hợp của 2 sự kiện A và B là sự kiện xảy ra khi và chỉ khi có ít nhất một trong hai sự kiện A hoặc B xảy ra hay \textbf{tổng sự kiện sơ cấp của A và B}\\
- Kí hiệu: $A \cup B$\\

\subsubsection[short]{Tích}
- Định nghĩa: Tích của 2 sự kiện A và B là sự kiện xảy khi và chỉ khi cả 2 sự kiện A và B đều xảy ra\\
- Kí hiệu: A.B

\subsubsection{Hiệu}
- Định nghĩa: là sự kiện xảy ra khi và chỉ khi A xảy ra nhưng B không xảy ra.\\
- Kí hiệu: $A \setminus B$\\

\subsubsection[short]{Xung khắc}
- Định nghĩa: nếu 2 sự kiện không đồng thời xảy ra hay \textbf{sự kiện sơ cấp của 2 sự kiện khác nhau nhưng hợp không bằng $\Omega$}\\
- Tức: $A.B=\emptyset$\\

\subsubsection{Đối lập}
- Định nghĩa: là sự kiện xảy ra khi và chỉ khi A không xảy ra hay \textbf{sự kiện sơ cấp của $\overline{A}= \Omega \setminus A$ và xung khắc}\\
- Kí hiệu: $\overline{A}$\\

\subsubsection{Nhóm đầy đủ các sự kiện}
- Định nghĩa: Các sự kiện $H_1, H_2, ..., H_n$ là nhóm đầy đủ các sự kiện nếu\\
+ Xung khắc đôi một, $H_i.H_j=\emptyset, i \neq j$\\
+ $H_1 \cup H_2 \cup ... H_n= \Omega$\\



\end{document}